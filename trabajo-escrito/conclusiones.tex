\renewcommand\thechapter{\Roman{chapter}}
\chapter{CONCLUSIONES} \label{ch:conclusiones} \thispagestyle{fancy}
\renewcommand\thechapter{\arabic{chapter}}
%%%%%%%%%%%%%%%%%%%%%%%%%%%%%%%%%%%%%%%%%%%%%%%%%%%%%%%%%%%%%%%%%%%%%%%%%%%%
En este trabajo de tesis no se logra terminar el simulador, pero se aprovecha el potencial que Matlab tiene, ya que no se había manejado anteriormente el programa para crear diseños virtuales. Como no se había utilizado no se conocían las funciones del todo y poco a poco se fueron viendo hasta lograr lo que se ve en los resultados, de igual forma falta integrar la programación para enlazar el panel con el manipulador y poder finalmente tener un simulador, para así poder compararlo con lo que se quiere llegar, que es similar pero en Java3D. Teniendo los cálculos en Matlab será mas fácil hacer la programación, estudiando la API de Java3D para ver el código de importación y tener control sobre las piezas diseñadas anteriormente en SolidWorks.\\

Este tipo de herramientas crean un impacto positivo ya que en los trabajos vistos dentro de la investigación son favorables y se espera llegar o superar esos resultados.\\

Lo interesante de este proyecto es que una vez terminado se le puede dar seguimiento ya que es  código abierto, en caso de querer añadir mas funciones será posible.