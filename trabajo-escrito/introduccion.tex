\renewcommand\thechapter{\Roman{chapter}}
\chapter{INTRODUCCIÓN} \label{ch:intro} \thispagestyle{fancy} \pagenumbering{arabic}
\renewcommand\thechapter{\arabic{chapter}}
%%%%%%%%%%%%%%%%%%%%%%%%%%%%%%%%%%%%%%%%%%%%%%%%%%%%%%%%%%%%%%%%%%%%%%%%%%%%
% trabajo cuenta con un diseño y desarrollo un laboratorio virtual el cual maneja un brazo robótico con el fin de que el usuario pueda interactuar y reforzar los conceptos básicos que se ven en el curso de robótica de nivel licenciatura. Es una herramienta la cual se enfoca en la ''enseñanza para la comprensión'', hoy en día los simuladores se vuelven mejor con las tecnologías que nos rodean, se pueden crear software el cual maneje mejor ciertos parámetros para asemejarse mas a la realidad que nos rodea y no necesitar de equipos caros, se busca igualar la manera de trabajar aunque no se tenga físicamente, pero que se sienta que la interacción es la misma.

%---------------------------------------------------------------------------------------
\section{Antecedentes}
Los simuladores surgen por la necesidad de crear sistemas de apoyo para el usuario ya sea aprendiendo de manera práctica, realizando descubrimientos, situaciones hipotéticas o simples pruebas de laboratorio aprovechando las especificaciones de la computadora para poder recrear dichas acciones visualizándolas de la mejor manera.\\

Hoy en día hay una infinidad de simuladores como análisis estadísticos, construcción, circuitos electrónicos, juegos y éstos ayudan a que las tareas sean más sencillas o que la teoría pueda aterrizarse de manera práctica de tal forma que se combinan muchas habilidades como el aprender a usar dicho programa, visualizar la manera en que se comporta la simulación y la aplicación de ésta misma; nos ayudan a evitar que ocurra un problema a la hora de aplicar el trabajo realizado esto es un factor importante, es evidente entonces que nos ahorra el tener que reparar un equipo en caso de que no haya funcionado, se observa claramente que otro punto es el hecho de hacernos saber si el proyecto es viable o no. En \cite{moran-2009} mencionan ''enseñanza para la comprensión'' que facilita el conocer los conceptos básicos de los robots creando una herramienta didáctica para ayudar en la comprensión de la teoría y logren obtener buenas bases sobres los conceptos manejados y las aplicaciones que tiene la robótica.\\

Existen formas las cuales pueden facilitar como lo son las pruebas físicas sin tener un equipo de una gran cantidad monetaria, se menciona en \cite{berenguel-2016} y \cite{lancheros-2010} que con piezas Lego$\copyright$  fueron capaces de realizar tele-robots mediante un software y un sistema electrónico el cual permitía aprender de una manera didáctica la forma en la que se comportaban los movimientos de dicho robot.\\

En la Universidad de Alicante \cite{pomares-2004} nos habla del impacto que tienen los laboratorios virtuales en los alumnos, dice que un 90\% de los alumnos necesita menos de 1 hora para poder comprender el simulador, mientras que a un 46\% le basta con 30 minutos, solo al 15\% les ha faltado tiempo para realizar la práctica propuesta con una duración máxima de 2 horas. Muestran gráficas las cuales concluyen que el 10\% de los alumnos necesitan más de 2 horas para poder comprender mejor el simulador, mientras que el 5\% lo dominan. Además, explican que muchos de los alumnos están conformes con trabajar vía simulación y desde casa ya que algunos asisten a la escuela y trabajan, pero el laboratorio virtual beneficia en el ámbito que les facilita el traslado y prefieren estar trabajando en su casa, por otra parte, el 71\% de los alumnos menciona que prefiere hacerlo en la escuela por las dudas que puedan surgir a la hora de estar haciendo las prácticas de dicho laboratorio.\\

En el libro \cite{selman-2002} nos habla de como el lenguaje de Java es uno de los cuales puede trasladarse de manera sencilla, el hecho de poder crear un código con una aplicación, ésta la podemos llevar ya sea en el celular, en la web, una PC con Windows, Mac OS o linux. La API de Java3D ofrece mucho al desarrollador, así como al definir objetos para elementos tales como apariencias, transformaciones, materiales, luces, etc.\\

El mundo de la robótica es caro, pero lo bueno es que existen programas alternos los cuales permiten crear, testear y simular de estos productos; Gazebo$\copyright$ \cite{gazebo-2014} es un simulador el cual permite al usuario usar un robot existente, modificar el mismo y adaptarle sensores para la aplicación final que se desee, es en tiempo real dicho lo anterior el poder usar sensores se encuentra dentro de los plugins del mismo programa ya que estos son las librerías con las cuales se pueden complementar el proyecto a realizar, la comunicación el ambiente de trabajo del robot, etc. Por otro lado, existe Coppelia Robotics$\copyright$ el cual cuenta con virtual robot experimentation platform (V-rep$\copyright$) es un programa portable ya que se puede manejar en Windows, Mac OS y linux.

\section{Definición del problema}
Como ejemplo tenemos que en las instituciones tratan siempre de tener tecnologías para los alumnos, equipos con los cuales puedan interactuar y aprender, pero en ciertas ocasiones existen impedimentos que los detienen para brindar ese servicio ya sea que el precio es muy elevado o que consideran que no vale la pena invertir para que solo tenga una aplicación básica y no se haga uso de todas las capacidades que éste tendría. En la clase de robótica el alumno muy difícilmente podría entender los conceptos teóricos, aunque estén todas las características de este, simplemente llevando a la práctica la teoría hay mas probabilidades de comprensión. 

¿Se puede realizar un simulador utilizando herramientas de software libre para el curso tradicional de robótica a nivel licenciatura?

\section{Justificación}
Un aspecto importante en la ingeniería es que a la hora de estar estudiando la teoría hay veces que se aterriza mejor esos conocimientos de manera práctica; Los laboratorios son importantes, es donde se aplican los conocimientos teóricos, pero no todos cuentan con equipos necesarios, ya que algunos pueden llegar a ser caros, por eso se busca que el usuario pueda interactuar por medio de estas herramientas y poder realizar acciones que complementen su clase para que comprenda mejor sus conocimientos, sin tener que tener un equipo físicamente, basta con un simulador que cumpla las características.

\section{Objetivos}
Diseñar y desarrollar un laboratorio virtual de robótica de manipuladores mediante la integración de gráficos vectorizados, cálculo simbólico y Java 3D con el fin de proporcionar una herramienta que fortalezca el proceso enseñanza-aprendizaje en el curso de robótica industrial.

\section{Hipótesis}
\begin{itemize}
\item Utilizar un software el cual permita diseñar en 3D un brazo robótico. 
\item Mediante Java3D poder recrear el diseño de Solidworks para que el usuario interactúe.
\end{itemize}

\section{Delimitaciones}
\begin{itemize}
\item El contenido aquí mencionado no se llevará a la práctica por lo que no se obtendrá una estadística para finalmente saber el impacto que este tendría. Es decir, no se probará el alcance de comprensión de la persona que se queda con los conceptos básicos de la clase y otra que adquiere conceptos de manera experimental en dicho laboratorio.
\item Se realizará un diseño en específico de un robot con 6 Grados De Libertad (GDL), esto indica que no podrá abarcar todos los posibles escenarios, distintas variantes de robots o de movimientos.
\end{itemize}

\section{Limitaciones}
\begin{itemize}
\item Debido a la falta de tiempo solo se llevará a cabo la implementación de la cinemática.
\end{itemize}
